

\documentclass{Report/thuemp}
\begin{document}

% 标题,作者
\emptitle{光学信息处理与傅里叶光学}
\empauthor{王驰}{王合英}

% 奇数页页眉 % 请在这里写出第一作者以及论文题目
\fancyhead[CO]{{\footnotesize 王驰: 光学信息处理与傅里叶光学}}

%%%%%%%%%%%%%%%%%%%%%%%%%%%%%%%%%%%%%%%%%%%%%%%%%%%%%%%%%%%%%%%%
% 关键词 摘要 首页脚注
%%%%%%%%关键词
\Keyword{光学信息处理,傅里叶光学,空间滤波,阿贝成像,衍射,卷积定理}
\twocolumn[
\begin{@twocolumnfalse}
\maketitle

%%%%%%%%摘要
\begin{empAbstract}
本实验利用基于 4F 结构的光学系统,系统研究了傅里叶光学的基本原理及其在信息处理中的应用。通过观察一维光栅($12\ \mathrm{line/mm}$)和二维正交光栅(“光”字屏)的夫琅禾费衍射图样,直观验证了会聚透镜的傅里叶变换性质,频谱面上的空间频率分布与物函数的周期和方向严格对应;利用自制滤波器(小孔、挡板、狭缝)实现了空间滤波,包括低通(图像模糊)、高通(边缘增强)及方向滤波(特定纹理提取);通过复合光栅滤波器($f_1=100\ \mathrm{line/mm}$, $f_2=102\ \mathrm{line/mm}$)观察到光学微分效应,突出了图像边缘;实验还通过两个卷积件的复合衍射,在频域验证了卷积定理;最后,利用 $\theta$ 调制片和白光光源实现了假彩色编码。实验结果直观地展示了在空间频域中处理信息的有效性,为光学计算和现代成像技术提供了实验基础。
\end{empAbstract}

%%%%%%%%英文标题、作者、摘要、关键词
\emptitleEn{Experiments of Modern Physics in Tsinghua University}
\empauthorEn{Chi Wang}{Heying Wang}
\KeywordEn{Optical Information Processing, Fourier Optics, Spatial Filtering, Abbe Imaging, Diffraction, Convolution Theorem}

\begin{empAbstractEn}
This experiment systematically investigated the fundamental principles of Fourier optics and its applications in information processing using a 4F-based optical system. The Fourier transform property of converging lenses was verified by observing the Fraunhofer diffraction patterns of a 1D grating ($12\ \mathrm{line/mm}$) and a 2D orthogonal grating ("Guang" character), showing strict correspondence between spatial frequency distribution and object properties. Spatial filtering, including low-pass (blurring), high-pass (edge enhancement), and directional filtering (texture extraction), was implemented using custom-made filters (pinholes, stops, slits). The optical differentiation effect, which enhances image edges, was observed using a composite grating filter ($f_1=100\ \mathrm{line/mm}$, $f_2=102\ \mathrm{line/mm}$). The convolution theorem was also experimentally verified in the frequency domain using composite diffraction elements. Finally, pseudo-color encoding was achieved through $\theta$ modulation using a white light source. The results vividly demonstrate the efficacy of processing information in the spatial frequency domain, providing an experimental foundation for optical computing and modern imaging technologies.
\end{empAbstractEn}

%%%%%%%%首页角注,依次为实验时间、报告时间、学号、email
\empfirstfoot{2025-05-18}{2025-05-25}{2022012259}{chi-wang22@mails.tsinghua.edu.cn}
\end{@twocolumnfalse}
]
%%%%%%%%!首页角注可能与正文重叠,请通过调整正文中第一页的\enlargethispage{-3.3cm}位置手动校准正文底部位置:
%%%%%%%%%%%%%%%%%%%%%%%%%%%%%%%%%%%%%%%%%%%%%%%%%%%%%%%%%%%%%%%%
%  正文由此开始
\wuhao 
%  分栏开始

\section{实验仪器}
\enlargethispage{-3.3cm}

本实验使用氦氖激光器( $\lambda=632.8\ \mathrm{nm}$ )作为相干光源,搭建如图 \ref{fig:setup} 所示的空间滤波实验光路。激光束经 $\mathrm{L_1}$ 扩束、$\mathrm{L_2}$ 准直后形成平行光,照射位于物面上的透明物体。$\mathrm{L_3}$ 为傅里叶变换透镜(焦距 $F=16.0\ \mathrm{cm}$ ),其后焦面 F 即为频谱面,用于放置各种滤波器。像面位于与物面关于 $L_3$ 共轭的位置,此处放置一电荷耦合装置(Charge Coupled Device, CCD)用于观察和记录滤波后的图像。这一组装置被用于空间成分滤波实验和方向滤波实验。

在 $\mathrm{L_3}$ 后方距离 $2F = 320 \ \mathrm{cm}$ 处,补充放置一焦距同为$F$的透镜 $\mathrm{L_4} $,并选择 $\mathrm{L_4}$ 后方距离$F$处作为像面,由此构成一个 $4F$ 系统。这一组装置被用于完成验证卷积定理、光学微分,以及$\theta$调制实验。

在以上实验装置中,透镜 $\mathrm{L_3}$ 均为在光学上实现光场傅里叶变换的核心器件。在薄透镜近似下,透镜作用相当于为从不同位置透过透镜的光添加一个额外相位$\Delta \varphi$。取图【】所示坐标系,考察一束平行光入射后会聚到焦点上的情形。根据费马(Fermat)原理,可以写出

\begin{align}
    \Delta \varphi (\xi, \eta) + k \sqrt{\xi^2+\eta^2+F^2} = \text{Const.}
\end{align}

傍轴近似条件下,定取相位基准$\Delta \varphi (\xi=0, \eta=0) = 0$,可以近似写出:

\begin{align}
    \Delta \varphi (\xi, \eta) = -k \frac{\xi^2+\eta^2}{2F}
\end{align}

现在考虑物面上的光场分布 $g(x,y)$ 与 $\mathrm{L_3}$ 的后焦面上光场分布 $G(x', y')$ 的关系:利用惠更斯(Huygens)原理,考察通过透镜后的子波叠加,可以写出:

\begin{align}
            &   % Split 01
            G(x', y')
        \notag \\
            \propto
            &   % Split 01   
            \iint
            \mathrm{d}x \mathrm{d}y \ g(x,y)
            \times \exp \Bigl\{
                -i \Delta \varphi(\xi, \eta)
            \Bigr.
        \notag \\
            &
            \quad
                - ik \sqrt{(x-\xi)^2+(y-\eta)^2+F^2}
        \notag \\
            &   % Split 01
            \quad \Bigl.
                - ik \sqrt{(x'-\xi)^2+(y'-\eta)^2+F^2}
            \Bigr\}
\end{align}

在傍轴近似下,假定透镜大小足够大,展开各平方根项并忽略高阶项,可以得到:

\begin{align}
        &
        G(x', y')
    \notag \\
        \propto
        &
        \iint \mathrm{d}x \mathrm{d}y \mathrm{d}\xi \mathrm{d}\eta \, g(x,y)
    \notag \\
        & 
        \times \exp \Bigl\{
            -   ik \frac{ (x - \xi) ^ 2 + (y - \eta) ^ 2 }{2F}
            +   ik \frac{ \xi ^ 2 + \eta ^ 2 }{2F}
        \Bigr.
    \notag \\
        &
        \quad \quad \Bigl.
            -   ik \frac{ (x' - \xi) ^ 2 + (y' - \eta) ^ 2}{2F}
            -   2ikF
        \Bigr\}
    \notag \\
        =
        &
        \iint \mathrm{d}x \mathrm{d}y \mathrm{d}\xi \mathrm{d}\eta \, g(x,y) \times \exp\bigl( -2ikF \bigr)
    \notag \\
        &
        \times \exp \Bigl\{
            -   ik \frac{x^2+y^2+x'^2+y'^2+\xi^2+\eta^2}{2F}
        \Bigr.
    \notag \\
        &
        \quad \quad \Bigl.
            +   ik \frac{(x+x')\xi + (y+y')\eta}{F}
        \Bigr\}
    \notag \\
        =
        &
        \iint \mathrm{d}x \mathrm{d}y \mathrm{d}\xi \mathrm{d}\eta \, g(x,y) \times \exp\bigl(-2ikF\bigr)
    \notag \\
        &
        \times \exp \Bigl\{
            +   ik \frac{xx'+yy'}{2F}
        \Bigr.
    \notag \\
        &
        \quad \quad \Bigl.
            -   ik \frac{
                \bigl[\xi  - (x+x')\bigr]^2
            +   \bigl[\eta - (y+y')\bigr]^2
            }{2F}
        \Bigr\}
    \notag \\
        \propto
        &
        \iint \mathrm{d}x \mathrm{d}y \, g(x,y) \exp \Bigl\{
            +   ik \frac{xx'+yy'}{2F}
        \Bigr\}
\end{align}

这就表明,在透镜 $\mathrm{L_3}$ 的后焦面上,光场分布 $G(x', y')$ 与物面光场分布 $g(x,y)$ 之间满足傅里叶变换关系:

\begin{align}
    G(x', y') \propto \mathcal{F}\bigl\{g(x,y)\bigr\} \Bigl(f_x = \frac{x'}{\lambda F}, f_y = \frac{y'}{\lambda F}\Bigr)
\end{align}

其中参数$f_x, f_y$分别为物面光场沿$x$和$y$方向的空间频率。由此,透镜 $\mathrm{L_3}$ 的后焦面被称为频谱面。

以上述傅里叶变换关系为基础,开展以下各项实验。

\section{实验内容}
 

\subsection{空间成分滤波}
使用一维光栅($10\ \textrm{线/mm}$),条纹沿竖直方向放置,构建物面光场。在频谱面 $F$ 处预期可观察到水平排列的衍射图样,也即物面光场的空间频率分布。

使用硬纸打孔,按下列情况分别允许特定空间频率成分通过,在像面借助CCD观察并记录条纹间距和形貌的变化,进行比较:

\begin{itemize}
    \item A组:分别通过 (0, $\pm 1$ 级)、(0, $\pm 2$ 级),以及不加滤波器;
    \item B组:分别通过 (0, +1 级)、(0, $\pm 1$, $\pm 2$ 级),以及不加滤波器;
    \item C组:分别通过 (仅 0 级)、(除 0 级外所有),以及不加滤波器。
\end{itemize}

其中,A组和B组用于分析不同频率成分对图像周期和形貌的影响,C组用于分析零频分量对图像的影响。

\subsection{方向滤波}
在物面放置正交光栅(“光”字屏),观察其二维点阵频谱。同样利用硬纸开孔,分别只允许以下方向的频谱分量通过:

\begin{itemize}
    \item 竖直方向(即只通过频谱的水平分量);
    \item 水平方向(即只通过频谱的竖直分量);
    \item $\pm 45^\circ$ 斜方向。
\end{itemize}

继续在像面借助CCD观察并记录像面图像的变化,分析频谱方向与图像特征方向的关系;进而使用图像分析软件,测量水平条纹与$45^\circ$斜方向条纹的空间周期之比。

\subsection{验证卷积定理}

傅里叶变换的重要性质之一为卷积定理。对于函数 $g_1(x,y)$ 和 $g_2(x,y)$,其卷积定义为:

\begin{align}
    \bigl(g_1 \ast g_2\bigr) (x,y) = \iint \mathrm{d}x' \mathrm{d}y' \ g_1(x', y') g_2(x-x', y-y')
\end{align}

卷积定理指出,函数的卷积在频域中对应于各自傅里叶变换的乘积:

\begin{align}
    \mathcal{F}\bigl\{g_1 \ast g_2\bigr\} = \mathcal{F}\bigl\{g_1\bigr\} \cdot \mathcal{F}\bigl\{g_2\bigr\}
\end{align}

使用图【】所示的 $4F$ 系统,在频谱面上单独放置透明卷积件时,其振幅透光率函数 $G_i(\xi,\eta) \quad (i = 1, 2) $ 的傅里叶变换即为物面光场 $g_i(x,y)$。若将两个卷积件依次密接放置在频谱面上,可观察像面图像 $g(x,y)$,验证其是否等于各自物面光场的卷积,进而达到验证卷积定理的目的。

考虑到卷积运算的难度,本部分实验中选择两块二维光栅作为卷积件,以简化验证过程;通过分别转动两块光栅,实现对两个原函数 $g_1(x,y)$ 和 $g_2(x,y)$ 的改变,以更充分地验证卷积定理。

\subsection{光学微分}

对于正弦光栅,其振幅透光率写成$ H(\xi, \eta) = H_0 + H_1\cos (2\pi f_0 \xi + \varphi_0)$ ,其傅里叶变换为:

\begin{align}
    h_0(x', y') = H_0 \delta(x', y') + \frac{H_1}{2} \Bigl[
        \delta\left(x' - \frac{f_0}{\lambda F}, y'\right) e^{i\varphi_0}
    +   \delta\left(x' + \frac{f_0}{\lambda F}, y'\right) e^{-i\varphi_0}
    \Bigr]
\end{align}

引入复合光栅,其振幅透光率中包含两个频率成分:
\begin{align}
    H(\xi, \eta) = H_0 + H_1\cos (2\pi f_1 \xi + \varphi_1) + H_2\cos (2\pi f_2 \xi + \varphi_2)
\end{align}

则其傅里叶变换为:
\begin{align}
    h(x', y') =& H_0 \delta(x', y') + \frac{H_1}{2} \Bigl[
        \delta\left(x' - \frac{f_1}{\lambda F}, y'\right) e^{i\varphi_1}
    +   \delta\left(x' + \frac{f_1}{\lambda F}, y'\right) e^{-i\varphi_1}
    \Bigr] \\
    &+ \frac{H_2}{2} \Bigl[
        \delta\left(x' - \frac{f_2}{\lambda F}, y'\right) e^{i\varphi_2}
    +   \delta\left(x' + \frac{f_2}{\lambda F}, y'\right) e^{-i\varphi_2}
    \Bigr]
\end{align}

考察复合光栅的两个频率成分各自给出的$+1$级像(以$x' = \frac{f_{1,2}}{\lambda F}$为中心);借助卷积定理得到,对应的像面光场为:
\begin{align}
g_{+1}(x', y') = \frac{H_1}{2} g\left(x' - \frac{f_1}{\lambda F}, y'\right) e^{i\varphi_1} + \frac{H_2}{2} g\left(x' - \frac{f_2}{\lambda F}, y'\right) e^{i\varphi_2}
\end{align} 

若设$f_2 = f_1 + \Delta f$,并假定$\Delta f$较小,则可以对上述式子进行泰勒展开,得到:
\begin{align}
    g\left(x' - \frac{f_2}{\lambda F}, y'\right)
    &= g\left(x' - \frac{f_1 + \Delta f}{\lambda F}, y'\right) \\
    &\approx g\left(x' - \frac{f_1}{\lambda F}, y'\right)
    - \frac{\Delta f}{\lambda F} \frac{\partial g}{\partial x'} \Bigg|_{x' - \frac{f_1}{\lambda F}, y'}
\end{align}
代入上式,得到:
\begin{align}
    g_{+1}(x', y')
    &\approx \left( \frac{H_1}{2} e^{i\varphi_1} + \frac{H_2}{2} e^{i\varphi_2} \right) g\left(x' - \frac{f_1}{\lambda F}, y'\right) \\
    &\quad - \frac{H_2}{2} \frac{\Delta f}{\lambda F} e^{i\varphi_2} \frac{\partial g}{\partial x'} \Bigg|_{x' - \frac{f_1}{\lambda F}, y'}
\end{align}

本实验中,在物面放置以图片(形状如图【】所示),在频谱面 F 处放置复合正弦光栅($f_1=100\ \textrm{线/mm}$, $f_2=102\ \textrm{线/mm}$)作为空间滤波器,在像面 $\pm 1$ 级衍射像处,预期可观察到图像的微分效果。垂直于光轴缓慢平移复合光栅,寻找并记录使 $\pm 1$ 级像的边缘轮廓最清晰的位置,拍摄记录。

\subsection{\texorpdfstring{$\theta$}{theta} 调制}
$\theta$片由多块不同取向的一维光栅拼成。使用白光入射,频谱面上依各个光栅取向形成一系列不同颜色的衍射光斑阵列。在频谱面上叠加孔阵滤波器,在适当的方向上保留适当颜色的$\pm 1$级衍射斑。在像面上,将能够得到一幅彩色图像。

本实验中,$\theta$片为一块三个取向光栅拼成的风景图样,三个光栅取向两两互成 $60^\circ$ 角,空间频率均为 $100\ \textrm{线/mm}$。同样使用硬纸打孔制作滤波器,在频谱面上仅允许三个方向的 $\pm 1$ 级衍射斑通过,在像面上观察并记录彩色图像。

\section{实验结果与分析}


%%%%%%%%%%%%%%%%%%%%%%%%%%%%%%%%%%%%%%%%%%%%%%%%%%%%%%%%%%%%%%%%
%  参考文献
%%%%%%%%%%%%%%%%%%%%%%%%%%%%%%%%%%%%%%%%%%%%%%%%%%%%%%%%%%%%%%%%
%  参考文献按GB/T 7714-2015《文后参考文献著录规则》的要求著录. 
%  参考文献在正文中的引用方法:\cite{bib文件条目的第一行}

\renewcommand\refname{\heiti\wuhao\centerline{参考文献}\global\def\refname{参考文献}}
\vskip 12pt


\let\OLDthebibliography\thebibliography
\renewcommand\thebibliography[1]{
  \OLDthebibliography{#1}
  \setlength{\parskip}{0pt}
  \setlength{\itemsep}{0pt plus 0.3ex}
}

{
\renewcommand{\baselinestretch}{0.9}
\liuhao
\bibliographystyle{gbt7714-numerical}
\bibliography{./Report/report}
}

\appendix
\section{部分实验数据}

\begin{table}[H]
    \centering
    \captionnamefont{\wuhao\bf\heiti}
    \captiontitlefont{\wuhao\bf\heiti}
    \caption{空间成分滤波(一维光栅)实验数据记录} \label{tab:data_1d_filter}
    \liuhao
    \begin{tabular}{c p{5cm} p{5cm}}
        \toprule
        滤波情况(通过的衍射级) & 像面图像特征(定性描述) & 备注(条纹周期等) \\
        \midrule
        0, $\pm 1$ 级 & 条纹清晰,边缘较模糊 & 周期 $d \approx 1/f_0$ \\
        0, $\pm 2$ 级 & 条纹清晰,边缘较模糊 & 周期 $d' \approx 1/(2f_0)$ \\
        0, $\pm 1$, $\pm 2$ 级 & 条纹清晰,边缘较锐利 & 周期 $d \approx 1/f_0$ \\
        仅 0 级 & 均匀亮背景,无条纹 & / \\
        $\pm 1$ 级(滤除 0 级) & 条纹清晰,背景暗,对比度反转 & 周期 $d' \approx 1/(2f_0)$ \\
        全部通过 & 条纹清晰,边缘锐利,接近方波 & 周期 $d \approx 1/f_0$ \\
        \bottomrule
    \end{tabular}
\end{table}

\begin{table}[H]
    \centering
    \captionnamefont{\wuhao\bf\heiti}
    \captiontitlefont{\wuhao\bf\heiti}
    \caption{方向滤波(“光”字屏)实验数据记录} \label{tab:data_2d_filter}
    \liuhao
    \begin{tabular}{c p{5cm} p{5cm}}
        \toprule
        滤波器类型 & 像面图像特征(定性描述) & 分析 \\
        \midrule
        无滤波器 & 清晰的“光”字 & 包含所有频率成分 \\
        竖直狭缝(过 0 级) & 仅显示“光”字的竖直笔画 & 滤除了竖直方向的频谱(对应水平笔画) \\
        水平狭缝(过 0 级) & 仅显示“光”字的水平笔画 & 滤除了水平方向的频谱(对应竖直笔画) \\
        45$^\circ$ 狭缝(过 0 级) & 未观察到清晰的特定方向笔画 & “光”字屏主要由水平和竖直结构构成 \\
        \bottomrule
    \end{tabular}
\end{table}


\end{document}