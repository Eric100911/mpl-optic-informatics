\documentclass[10pt]{beamer}

\usepackage[utf8]{inputenc}
\usepackage{ctex} % For Chinese characters support
\usepackage{amsmath}
\usepackage{graphicx}
\usepackage{booktabs} % For nice tables

% Beamer Theme
\usetheme{Madrid}
\usecolortheme{default}

% Title Page Information
\title[光学信息处理实验]{近代物理实验:光学信息处理}
\author{王驰 2022012259}
\institute{致理书院}
\date{\today}

\begin{document}

% Title Page Frame
\begin{frame}
    \titlepage
\end{frame}

% Outline Frame
\begin{frame}{报告提纲}
    \tableofcontents
\end{frame}

% Section 1: Introduction

\section{引言}

\begin{frame}{Fourier变换}
    \begin{block}{傅里叶变换的基本概念}
        傅里叶变换 (Fourier Transform, FT) 是一种数学工具,用于将信号从空间域 (Spatial Domain) 转换到频域 (Frequency Domain)。
        \begin{itemize}
            \item \textbf{空间域}: 信号在 $(x,y)$ 坐标平面的表示,如图像等。
            \item \textbf{频域}: 信号由不同频率成分组成的表示,揭示了信号的频率结构。
        \end{itemize}
    \end{block}
    \begin{alertblock}{傅里叶变换公式}
        \begin{equation}
        G(f_x, f_y) = \int_{-\infty}^{\infty} \int_{-\infty}^{\infty} g(x,y) e^{-2\pi i (f_x x + f_y y)} \, dx \, dy
        \end{equation}
        其中,$g(x,y)$ 是空间域信号,$G(f_x, f_y)$ 是对应的频域信号。
    \end{alertblock}
\end{frame}

\begin{frame}{Fourier变换的卷积定理}
    \begin{block}{卷积定理}
        傅里叶变换的卷积定理指出,两个函数的卷积在频域中对应于它们傅里叶变换的乘积:
        \begin{equation}
        \mathcal{F}\{g(x,y) * h(x,y)\} = G(f_x, f_y) \cdot H(f_x, f_y)
        \end{equation}
        其中,$*$ 表示卷积运算,$G(f_x, f_y)$ 和 $H(f_x, f_y)$ 分别是 $g(x,y)$ 和 $h(x,y)$ 的傅里叶变换。
    \end{block}
    \begin{alertblock}{物理意义}
        这意味着在频域中进行滤波操作(如低通、高通滤波)相当于在空间域中对信号进行卷积,从而实现对图像的各种处理效果。
    \end{alertblock}    
\end{frame}

% - How lenses perform Fourier transforms

\begin{frame}{透镜的Fourier变换性质}
    对薄透镜,傍轴近似下,透镜的作用是给入射波前添加一个二次相位因子:
    \begin{equation}
    \tilde{T}_L(x,y) = \exp\left(-ik\frac{x^2+y^2}{2F}\right)
    \end{equation}
    其中,$F$ 是透镜的焦距,$k = \frac{2\pi}{\lambda}$ 是波数。
    当相干平行光照射位于透镜前焦面的物体时,在透镜的后焦面上得到的光场分布$G(x', y')$,恰正比于物光场$g(x,y)$的傅里叶变换:
    \begin{equation}
    G(x', y') \propto \int_{-\infty}^{\infty} \int_{-\infty}^{\infty} g(x,y) e^{-2\pi i (f_x x + f_y y)} \, dx \, dy
    \end{equation}
    其中,频率坐标与空间坐标的关系为:
    \begin{equation}
    f_x = \frac{x'}{\lambda F}, \quad f_y = \frac{y'}{\lambda F}
    \end{equation}
    这表明透镜可以实现空间域到频域的转换。
\end{frame}

\begin{frame}{Abbe成像原理与空间滤波}
    \begin{block}{Abbe成像原理}
        Abbe成像原理指出,物体的每个空间频率分量都可以看作是一个独立的波源,这些波源在成像系统中干涉叠加形成最终的像。
        \begin{itemize}
            \item 物镜对物体进行傅里叶变换,在频谱面形成频谱。
            \item 频谱面上的各频率分量作为新波源,干涉叠加形成像。
        \end{itemize}
    \end{block}
    \begin{alertblock}{空间滤波}
        在频谱面上放置滤波器,可以选择性地修改频谱分量,从而改变最终的像。
        \begin{itemize}
            \item \textbf{低通滤波}: 去除高频分量,模糊图像。
            \item \textbf{高通滤波}: 去除低频分量,锐化边缘。
            \item \textbf{方向滤波}: 提取特定方向的纹理信息。
        \end{itemize}
    \end{alertblock}
\end{frame}

\begin{frame}{4F系统}
    \begin{figure}
        \includegraphics[width=0.8\textwidth]{Prep/4F-system.png} % Replace with actual diagram if available
        \caption{4F系统示意图\cite{zhaoOptics2004_4F}}
        \label{fig:4F-system}
    \end{figure}

    4F系统由两个焦距为$F$的透镜组成,物体位于第一个透镜的前焦面,像位于第二个透镜的后焦面。频谱面位于两个透镜的公共焦平面上。
\end{frame}

\begin{frame}{复合光栅的Fourier变换性质}
    在频谱面上放置正弦光栅作为滤波器,其透光率满足:
    \begin{equation}
        H(\xi, \eta) = H_0 + H_1\cos (2\pi f_0 \xi + \varphi_0)
    \end{equation}
    其本身Fourier变换结果为:
    \begin{equation}
        h_0(x', y') = H_0 \delta(x', y') + \frac{H_1}{2} \left[ \delta\left(x' - \frac{f_0}{\lambda F}, y'\right) e^{i\varphi_0} + \delta\left(x' + \frac{f_0}{\lambda F}, y'\right) e^{-i\varphi_0} \right]
    \end{equation}
    若引入复合光栅作为滤波器,具有透光率分布:
    \begin{equation}
        H(\xi, \eta) = H_0 + H_1\cos (2\pi f_1 \xi + \varphi_1) + H_2\cos (2\pi f_2 \xi + \varphi_2)
    \end{equation}
    则其Fourier变换结果为:
    \begin{align}
        h(x', y') =& H_0 \delta(x', y') + \frac{H_1}{2} \left[ \delta\left(x' - \frac{f_1}{\lambda F}, y'\right) e^{i\varphi_1} + \delta\left(x' + \frac{f_1}{\lambda F}, y'\right) e^{-i\varphi_1} \right] \\
        &+ \frac{H_2}{2} \left[ \delta\left(x' - \frac{f_2}{\lambda F}, y'\right) e^{i\varphi_2} + \delta\left(x' + \frac{f_2}{\lambda F}, y'\right) e^{-i\varphi_2} \right]
    \end{align}
\end{frame}

\begin{frame}{使用复合光栅进行微分运算}
    考察复合光栅的两个频率成分各自给出的$+1$级像(以$x' = \frac{f_{1,2}}{\lambda F}$为中心);借助卷积定理得到,对应的像面光场为:
    \begin{equation}
    g_{+1}(x', y') = \frac{H_1}{2} g\left(x' - \frac{f_1}{\lambda F}, y'\right) e^{i\varphi_1} + \frac{H_2}{2} g\left(x' - \frac{f_2}{\lambda F}, y'\right) e^{i\varphi_2}
    \end{equation}
    考虑$f_2 = f_1 + \Delta f$($\Delta f \ll f_1$)情形,选取$\varphi_2 = \varphi_1 + \pi$ 且 $H_1 = H_2$,则有:
    \begin{align}
        g_{+1}(x', y') &= \frac{H_1}{2} \left[ g\left(x' - \frac{f_1}{\lambda F}, y'\right) - g\left(x' - \frac{f_1 + \Delta f}{\lambda F}, y'\right) \right] e^{i\varphi_1} \\
        &\approx -\frac{H_1 \Delta f}{2 \lambda F} \frac{\partial g}{\partial x'}\left(x' - \frac{f_1}{\lambda F}, y'\right) e^{i\varphi_1}
    \end{align}
    这表明复合光栅实现了对图像的微分运算。
\end{frame}

\begin{frame}{$\theta$ 调制}
    \begin{figure}
        \includegraphics[width=0.6\textwidth]{Prep/theta-plane.png}
        \caption{$\theta$ 调制示意图\cite{zhaoOptics2004_4F}}
        \label{fig:theta-plane}
    \end{figure}
    $\theta$片由多块不同取向的一维光栅拼成。使用白光入射,频谱面上依各个光栅取向形成一系列不同颜色的衍射光斑阵列。在频谱面上叠加孔阵滤波器,在适当的方向上保留适当颜色的$\pm 1$级衍射斑。在像面上,将能够得到一幅彩色图像。
\end{frame}
% - In two columns, left side the progress, right side the image

\begin{frame}{光学信息处理的前沿方向}
    \begin{columns}[T]
        \begin{column}{0.4\textwidth}
            \begin{itemize}
                \item \textbf{超构表面 (Metasurfaces)}: 使用亚波长结构设计“平坦光学”元件,实现对光波的任意调控。
                \item \textbf{光学计算与神经网络}: 利用衍射光学实现神经网络中的核心运算——矩阵乘法\cite{zhouLargescaleNeuromorphicOptoelectronic2021},功耗远低于电子计算。
                \item \textbf{计算成像 (Computational Imaging)}: 结合光学硬件编码与软件算法解码,实现超越传统极限的成像。
            \end{itemize}
        \end{column}
        \begin{column}{0.6\textwidth}
            \begin{figure}
                \includegraphics[width=0.9\textwidth]{Prep/DPU-framework-Dai.png}
                \caption{使用衍射光学进行神经网络中矩阵计算的框架示意图\cite{zhouLargescaleNeuromorphicOptoelectronic2021}}
                \label{fig:DPU-framework-Dai}
            \end{figure}
        \end{column}
    \end{columns}
\end{frame}

\section{实验设计}

\begin{frame}{空间成分滤波}
    \begin{block}{实验目的}
        理解不同空间频率成分对图像(一维光栅)形态的贡献。通过滤波不同衍射级,观察像面图像的变化。
    \end{block}
    \begin{block}{实验步骤}
        \begin{enumerate}
            \item 在物面放置一维光栅(条纹竖直)。
            \item 在频谱面观察水平排列的衍射光斑。
            \item 用滤波器选择不同频谱分量通过。
            \item 在像面观察并记录滤波后的图像。
        \end{enumerate}
    \end{block}
\end{frame}

\begin{frame}{方向滤波}
    \begin{block}{实验目的}
        对二维图像(正交光栅)进行特定方向信息的提取。
    \end{block}
    \begin{block}{实验步骤}
        \begin{enumerate}
            \item 物面放置正交光栅(如“光”字屏)。
            \item 观察二维点阵频谱。
            \item 用狭缝状滤波器选择水平、竖直或$45^\circ$方向频谱。
            \item 观察像面图像,测算水平与$45^\circ$方向斜条纹的空间周期之比。
        \end{enumerate}
    \end{block}
\end{frame}

\begin{frame}{验证卷积定理}
    \begin{block}{实验目的}
        通过实验现象理解卷积定理:两个函数之积的傅里叶变换等于各自变换的卷积。
    \end{block}
    \begin{block}{实验步骤}
        \begin{enumerate}
            \item 分别观察卷积件1(如字母“A”)和卷积件2(如小孔)的频谱。
            \item 将两者重叠放置,观察叠加后的频谱。
        \end{enumerate}
    \end{block}
\end{frame}

\begin{frame}{使用4F系统进行$\theta$ 调制}
    \begin{block}{实验目的}
        实现图像边缘增强(光学微分),通过复合光栅滤波器在频谱面进行$\theta$调制。
    \end{block}
    \begin{block}{实验步骤}
        \begin{enumerate}
            \item 物面放置待微分图片。
            \item 频谱面放置复合光栅滤波器。
            \item 缓慢平移复合光栅,观察±1级像的变化。
            \item 记录边缘增强图像。
        \end{enumerate}
    \end{block}
\end{frame}

\section{注意事项}

\begin{frame}{注意事项}
    \begin{block}{关键注意事项}
        \begin{itemize}
            \item \textbf{激光安全}: 严禁眼睛直视光路!
            \item \textbf{元件清洁}: 避免触摸光学面。
            \item \textbf{共轴调节}: 保证光路精度是实验成功的前提 。
            \item \textbf{光束准直}: 使用平晶法,确保扩束后入射到物体的为平行光。
        \end{itemize}
    \end{block}
\end{frame}

\bibliographystyle{plain}
\bibliography{Prep/prep-slides}

\end{document}